%!TEX TS-program = xetex-xdvipdfmx
%!TEX encoding = UTF-8 Unicode

\input font-tests.tex
%\labelsfalse will turn off sample labels
%\labelstrue will turn them back on

\papersize{a4}
%\landscape will turn page
%\landscapefalse will undo it

\testfont{NokyungAC}

%\testfont{Dai Banna SIL Light}
%font features can be specified as part of the font name (see separate docs)

%\sizes={13/19/3.75in}

\testfont{Times New Roman}
\sizes={12}
\dotext A theory may be viewed rather broadly as a statement purporting to describe, or to explain, or to help one to understand a phenomenon. More narrowly, a theory may present a claim of truth, or assert the presence of relationships between phenomena, or predict the occurrence of phenomena. (Times New Roman)

\testfont{Cambria}
\sizes={12}
\dotext A theory may be viewed rather broadly as a statement purporting to describe, or to explain, or to help one to understand a phenomenon. More narrowly, a theory may present a claim of truth, or assert the presence of relationships between phenomena, or predict the occurrence of phenomena. (Cambria)

\testfont{Candara}
\sizes={12}
\dotext A theory may be viewed rather broadly as a statement purporting to describe, or to explain, or to help one to understand a phenomenon. More narrowly, a theory may present a claim of truth, or assert the presence of relationships between phenomena, or predict the occurrence of phenomena. (Candara)

\testfont{Arial}
\sizes={12}
\dotext A theory may be viewed rather broadly as a statement purporting to describe, or to explain, or to help one to understand a phenomenon. More narrowly, a theory may present a claim of truth, or assert the presence of relationships between phenomena, or predict the occurrence of phenomena. (Arial)

\testfont{NokyungAC}
\sizes={12}
\dotext ᦝᧂᦑᦸᦰ ᦍᦸᧆᦑᦲᧈᦷᦢᦆᧄ ᦅᧀᦂᦱᧂᦐᦸᧂ ᦂᦱᧁ ᦙᦸᧃᦟᦱᧆᦓᧄᧉ, ᦶᦙᧈᦷᦎᦶᦂᧄᧉ ᦃᦸᧂᧈ ᦋᧄᧉ ᦶᦀᧁ ᦵᦉᧄᧉ ᦺᦜᧈ ᦺᦃ ᦋᦻᦵᦣᧀ, ᦔᦱᧂ ᦓᦲᦰ ᦋᦻ ᦈᧅ ᦺᦃ ᦺᦔ ᦡᦽᧉᦓᦲᧉ ᦑᦱᧃᦜᧂ ᦶᦎᧈ ᦂᦸᧃᧈ, ᦈᦱ ᦉᦸᧃᧈ ᦠᦹᧉ ᦵᦗᦲᧃᧈᦣᦼᧉᦡᦽᧉᦆᦱᧁᧈᦟᧂᦂᦱ, ᦎᦱᧄᦔᦱᦟᦲᦀᦸᧅᧈ ᦒᧄᦟᧁᧈᦺᦞᧉᦗᦸᦡᧂᧈᦉᦸᧄᧉ ᦡᦸᧅᧈᦺᦙᧉ ᦠᦸᧄ ᦶᦕᧈ ᦵᦙᦲᧂᦺᦑ, ᦅᧄᦺᦘᦷᦎᧅᦶᦑᧃᦺᦡᦵᦗᦲᧃᧈᦠᦱᧅᦟᦹᦌᦰ ᦌᦱᧁᧉ, ᦆᦱᧁᧈᦠᦱᧅᦂᦲᧉᦍᦱᧁᦶᦑᧉ ᦵᦖᦲᧃᦡᧂᧈᦓᧄᧉ ᦶᦙᧈᦦᦱᧂᧉᦉᦖᦳᧆᦺᦊᧈᦉᦱᦅᦸᧃ, ᦢᧆᦓᦲᦰᦂᦸᧃᦅᧄᦙᦲ ᦷᦜᧂᦜᦻᦺᦉᧈ ᦺᦢ ᦑᧂ ᦉᦾᧉ,  (Nokyung AC)

\testfont{Dai Banna SIL Book}
\sizes={12}
\dotext ᦝᧂᦑᦸᦰ ᦍᦸᧆᦑᦲᧈᦷᦢᦆᧄ ᦅᧀᦂᦱᧂᦐᦸᧂ ᦂᦱᧁ ᦙᦸᧃᦟᦱᧆᦓᧄᧉ, ᦶᦙᧈᦷᦎᦶᦂᧄᧉ ᦃᦸᧂᧈ ᦋᧄᧉ ᦶᦀᧁ ᦵᦉᧄᧉ ᦺᦜᧈ ᦺᦃ ᦋᦻᦵᦣᧀ, ᦔᦱᧂ ᦓᦲᦰ ᦋᦻ ᦈᧅ ᦺᦃ ᦺᦔ ᦡᦽᧉᦓᦲᧉ ᦑᦱᧃᦜᧂ ᦶᦎᧈ ᦂᦸᧃᧈ, ᦈᦱ ᦉᦸᧃᧈ ᦠᦹᧉ ᦵᦗᦲᧃᧈᦣᦼᧉᦡᦽᧉᦆᦱᧁᧈᦟᧂᦂᦱ, ᦎᦱᧄᦔᦱᦟᦲᦀᦸᧅᧈ ᦒᧄᦟᧁᧈᦺᦞᧉᦗᦸᦡᧂᧈᦉᦸᧄᧉ ᦡᦸᧅᧈᦺᦙᧉ ᦠᦸᧄ ᦶᦕᧈ ᦵᦙᦲᧂᦺᦑ, ᦅᧄᦺᦘᦷᦎᧅᦶᦑᧃᦺᦡᦵᦗᦲᧃᧈᦠᦱᧅᦟᦹᦌᦰ ᦌᦱᧁᧉ, ᦆᦱᧁᧈᦠᦱᧅᦂᦲᧉᦍᦱᧁᦶᦑᧉ ᦵᦖᦲᧃᦡᧂᧈᦓᧄᧉ ᦶᦙᧈᦦᦱᧂᧉᦉᦖᦳᧆᦺᦊᧈᦉᦱᦅᦸᧃ, ᦢᧆᦓᦲᦰᦂᦸᧃᦅᧄᦙᦲ ᦷᦜᧂᦜᦻᦺᦉᧈ ᦺᦢ ᦑᧂ ᦉᦾᧉ,  (Dai Banna SIL Book)
\testfont{NokyungAC}

\vfill
\end
